\documentclass[dvipdfmx]{jsarticle}
\usepackage[dvipdfmx]{graphicx}
\usepackage{amsmath,amssymb,amsfonts, ascmac, mathcomp, textcomp, siunitx,pdfpages, float, physics}

\setlength{\textheight}{45\baselineskip}
\setlength{\textwidth}{15cm}
\setlength{\voffset}{-3\baselineskip}
\setlength{\oddsidemargin}{21pt}
\setlength{\evensidemargin}{21pt}

\begin{document}

\centerline{熱力学第2回・レポート課題 問題}
\medskip
\rightline{TeXコード https://github.com/tksnaka/physics/blob/main/thermodynamics/report1.tex}
\bigskip


\bigskip 1. (1) 二変数関数$f(x, y)$について, $\displaystyle\frac{\partial f}{\partial x}$, $\displaystyle\frac{\partial f}{\partial y}$の定義と意味を答えよ. 

(2) $f(x, y) = x^2 y + x y^3$のとき, $\displaystyle\frac{\partial f}{\partial x}$, $\displaystyle\frac{\partial f}{\partial y}$, $\displaystyle\frac{\partial ^2 f}{\partial x \partial y}$, $\displaystyle\frac{\partial ^2 f}{\partial y \partial x}$を求めよ. 


\bigskip  2. (1) $f(x, y) = \mathrm{const.}$のとき, 
\[
    \frac{\mathrm{d} y}{\mathrm{d} x} = - \frac{\left(\frac{\partial f}{\partial x}\right)}{\left(\frac{\partial f}{\partial y}\right)}
\]

を示せ. 

(2) $f(x, y) \equiv \mathrm{e}^{-a(x^2 + y^2)}\cos{(2 a l_x x + 2 a l_y y)} = \mathrm{const.}$のとき, $\displaystyle\frac{\partial f}{\partial x}$, $\displaystyle\frac{\partial f}{\partial y}$, $\displaystyle\frac{\mathrm{d} y}{\mathrm{d} x}$を求めよ. ただし, $a$, $l_x$, $l_y$は0でない定数. 

\bigskip 3. (1) $f(x, y, z) = \mathrm{const.}$のとき, $\displaystyle\frac{\partial y}{\partial x} \cdot \displaystyle\frac{\partial z}{\partial y} \cdot \displaystyle\frac{\partial x}{\partial z} = -1$であることを示せ. 

(2) $f(x, y, z) \equiv (x^2 + y^2 + z^2) \mathrm{e}^{-2(a_x x + a_y y + a_z z)} = \mathrm{const.}$のとき, $\displaystyle\frac{\partial y}{\partial x}$, $\displaystyle\frac{\partial z}{\partial y}$, $\displaystyle\frac{\partial x}{\partial z}$及び, 積$\displaystyle\frac{\partial y}{\partial x} \cdot \displaystyle\frac{\partial z}{\partial y} \cdot \displaystyle\frac{\partial x}{\partial z}$を求めよ. ただし, $a_x$, $a_y$, $a_z$は0でない定数. 

\newpage

\centerline{熱力学第2回・レポート課題 解答}
\medskip

\bigskip


\bigskip 1. (1) 二変数関数$f(x, y)$について, $\displaystyle\frac{\partial f}{\partial x}$, $\displaystyle\frac{\partial f}{\partial y}$は以下の式で定義される. 

\begin{align*}
    \frac{\partial f}{\partial x} &\equiv \lim_{h \to 0} \frac{f(x + h, y) - f(x, y)}{h} \\
    \frac{\partial f}{\partial y} &\equiv \lim_{h \to 0} \frac{f(x, y + h) - f(x, y)}{h} 
\end{align*}

$\displaystyle\frac{\partial f}{\partial x}$は$f(x, y)$において$y$を固定した時の$x$の微分,  $\displaystyle\frac{\partial f}{\partial y}$は$x$を固定した時の$y$の微分を意味する. 

(2) $f(x, y) = x^2 y + x y^3$のとき, 

\begin{align*}
    \frac{\partial f}{\partial x} &= 2 x y + y^3 \\
    \frac{\partial f}{\partial y} &= x^2 + 3 x y^2 \\
    \frac{\partial ^2 f}{\partial x \partial y} &= \frac{\partial}{\partial x} \frac{\partial f}{\partial y} = \frac{\partial}{\partial x} (x^2 + 3 x y^2) = 2 x + 3 y^2 \\
    \frac{\partial ^2 f}{\partial y \partial x} &= \frac{\partial}{\partial y} \frac{\partial f}{\partial x} = \frac{\partial}{\partial y} (2 x y + y^3) = 2 x + 3 y^2 
\end{align*}

\bigskip 2. (1) $f(x, y) = \mathrm{const.}$より, $f$の完全微分は0であるから, 
\begin{equation*}
    \mathrm{d}f = \left(\frac{\partial{f}}{\partial x}\right) \mathrm{d}x + \left(\frac{\partial{f}}{\partial y}\right) \mathrm{d}y = 0
\end{equation*}

よって, 微小量$\mathrm{d}x$, $\mathrm{d}y$を限りなく0に近づけると, 

\begin{equation*}
    \frac{\mathrm{d}y}{\mathrm{d}x} = \lim_{\mathrm{d}x, \mathrm{d}y \to 0} \frac{\mathrm{d}y}{\mathrm{d}x} = - \frac{\left(\frac{\partial{f}}{\partial{x}}\right)} {\left(\frac{\partial{f}}{\partial{y}}\right)}
\end{equation*}

(2) $f(x, y) \equiv \mathrm{e}^{-a(x^2 + y^2)}\cos{(2 a l_x x + 2 a l_y y)} = \mathrm{const.}$のとき, 

\begin{align*}
    \frac{\partial f}{\partial x} &= \left\{-2ax \mathrm{e}^{-a(x^2 + y^2)} \cos{(2 a l_x x + 2 a l_y y)} \right\} + \left\{-2a l_x \mathrm{e}^{-a(x^2 + y^2)} \sin{(2 a l_x x + 2 a l_y y)} \right\} \\
    &= -2a\mathrm{e}^{-a(x^2 + y^2)} \left\{x \cos{(2 a l_x x + 2 a l_y y)} + l_x \sin{(2 a l_x x + 2 a l_y y)} \right\} \\
    \frac{\partial f}{\partial y} &= \left\{-2ay \mathrm{e}^{-a(x^2 + y^2)} \cos{(2 a l_x x + 2 a l_y y)} \right\} + \left\{-2a l_y \mathrm{e}^{-a(x^2 + y^2)} \sin{(2 a l_x x + 2 a l_y y)} \right\} \\
    &= -2a\mathrm{e}^{-a(x^2 + y^2)} \left\{y \cos{(2 a l_x x + 2 a l_y y)} + l_y \sin{(2 a l_x x + 2 a l_y y)} \right\} \\
    \frac{\mathrm{d}y}{\mathrm{d}x} &= - \frac{\left(\frac{\partial{f}}{\partial{x}}\right)} {\left(\frac{\partial{f}}{\partial{y}}\right)} \\
    &= - \frac{-2a\mathrm{e}^{-a(x^2 + y^2)} \left\{x \cos{(2 a l_x x + 2 a l_y y)} + l_x \sin{(2 a l_x x + 2 a l_y y)} \right\}}{-2a\mathrm{e}^{-a(x^2 + y^2)} \left\{y \cos{(2 a l_x x + 2 a l_y y)} + l_y \sin{(2 a l_x x + 2 a l_y y)} \right\}} \\
    &= \frac{x + l_x \tan{(2 a l_x x + 2 a l_y y)}}{y + l_y \tan{(2 a l_x x + 2 a l_y y)}}
\end{align*}

\bigskip 3. (1) $f(x, y, z) = \mathrm{const.}$ のとき, 

\begin{equation*}
    \mathrm{d}f = \left(\frac{\partial{f}}{\partial x}\right) \mathrm{d}x + \left(\frac{\partial{f}}{\partial y}\right) \mathrm{d}y + \left(\frac{\partial{f}}{\partial z}\right) \mathrm{d}z = 0
\end{equation*}

$z$を固定すると, $\mathrm{d}z = 0$. よって, 

\begin{equation*}
    \mathrm{d}f = \left(\frac{\partial{f}}{\partial x}\right) \mathrm{d}x + \left(\frac{\partial{f}}{\partial y}\right) \mathrm{d}y = 0
\end{equation*}

これより, 

\begin{equation*}
    \frac{\mathrm{d} y}{\mathrm{d} x} = - \frac{\left(\frac{\partial f}{\partial x}\right)}{\left(\frac{\partial f}{\partial y}\right)}
\end{equation*}

である. この時, 微小量$\mathrm{d}x$, $\mathrm{d}y$を限りなく0に近づけると, 

\begin{equation*}
    \frac{\partial y}{\partial x} = \lim_{\mathrm{d}x, \mathrm{d}y \to 0} \frac{\mathrm{d} y}{\mathrm{d} x} = - \frac{\left(\frac{\partial f}{\partial x}\right)}{\left(\frac{\partial f}{\partial y}\right)}
\end{equation*}

これと同様にして, 

\begin{align*}
    \frac{\partial z}{\partial y} &= - \frac{\left(\frac{\partial f}{\partial y}\right)}{\left(\frac{\partial f}{\partial z}\right)} \\
    \frac{\partial x}{\partial z} &= - \frac{\left(\frac{\partial f}{\partial z}\right)}{\left(\frac{\partial f}{\partial x}\right)}
\end{align*}

である. 以上より, 

\begin{align*}
    \frac{\partial y}{\partial x} \cdot \frac{\partial z}{\partial y} \cdot \frac{\partial x}{\partial z} = \left\{- \frac{\left(\frac{\partial f}{\partial x}\right)}{\left(\frac{\partial f}{\partial y}\right)}\right\} \cdot \left\{-\frac{\left(\frac{\partial f}{\partial y}\right)}{\left(\frac{\partial f}{\partial z}\right)}\right\} \cdot \left\{-\frac{\left(\frac{\partial f}{\partial z}\right)}{\left(\frac{\partial f}{\partial x}\right)}\right\} = -1
\end{align*}


(2) $f(x, y, z) \equiv (x^2 + y^2 + z^2) \mathrm{e}^{-2(a_x x + a_y y + a_z z)} = \mathrm{const.}$のとき, 

\begin{align*}
    \frac{\partial f}{\partial x} &= 2x\mathrm{e}^{-2(a_x x + a_y y + a_z z)} -2a_x(x^2 + y^2 + z^2)\mathrm{e}^{-2(a_x x + a_y y + a_z z)} \\
    &= \{2x - 2a_x(x^2 + y^2 + z^2)\}\mathrm{e}^{-2(a_x x + a_y y + a_z z)} \\
    \frac{\partial f}{\partial y} &= 2y\mathrm{e}^{-2(a_x x + a_y y + a_z z)} -2a_y(x^2 + y^2 + z^2)\mathrm{e}^{-2(a_x x + a_y y + a_z z)} \\
    &= \{2y - 2a_y(x^2 + y^2 + z^2)\}\mathrm{e}^{-2(a_x x + a_y y + a_z z)} \\
    \frac{\partial f}{\partial z} &= 2z\mathrm{e}^{-2(a_x x + a_y y + a_z z)} -2a_z(x^2 + y^2 + z^2)\mathrm{e}^{-2(a_x x + a_y y + a_z z)} \\
    &= \{2z - 2a_z(x^2 + y^2 + z^2)\}\mathrm{e}^{-2(a_x x + a_y y + a_z z)} 
\end{align*}

であるから, 

\begin{align*}
    \frac{\partial y}{\partial x} &= - \frac{\left(\frac{\partial f}{\partial x}\right)}{\left(\frac{\partial f}{\partial y}\right)} = - \frac{2x - 2a_x(x^2 + y^2 + z^2)}{2y - 2a_y(x^2 + y^2 + z^2)} \\
    \frac{\partial z}{\partial y} &= - \frac{\left(\frac{\partial f}{\partial y}\right)}{\left(\frac{\partial f}{\partial z}\right)} = - \frac{2y - 2a_y(x^2 + y^2 + z^2)}{2z - 2a_z(x^2 + y^2 + z^2)} \\
    \frac{\partial x}{\partial z} &= - \frac{\left(\frac{\partial f}{\partial z}\right)}{\left(\frac{\partial f}{\partial x}\right)} = - \frac{2z - 2a_z(x^2 + y^2 + z^2)}{2x - 2a_x(x^2 + y^2 + z^2)} 
\end{align*}

である. 以上より, 

\begin{align*}
    \frac{\partial y}{\partial x} \cdot \frac{\partial z}{\partial y} \cdot \frac{\partial x}{\partial z} &=  \left\{- \frac{2x - 2a_x(x^2 + y^2 + z^2)}{2y - 2a_y(x^2 + y^2 + z^2)} \right\} \cdot \left\{- \frac{2y - 2a_y(x^2 + y^2 + z^2)}{2z - 2a_z(x^2 + y^2 + z^2)}\right\} \cdot \left\{- \frac{2z - 2a_z(x^2 + y^2 + z^2)}{2x - 2a_x(x^2 + y^2 + z^2)} \right\} \\
    &= -1
\end{align*}

\end{document}
